% &%%%%%%%%%%%%%%%%%%%%%%%%%%%%%%%%%%%%%%%%%%%%%%%%%%%%%%%%%%%
% &%%                                                       %%
% &%%               MANUSCRIT DE THÈSE                      %%
% &%%                                                       %%
% &%%%%%%%%%%%%%%%%%%%%%%%%%%%%%%%%%%%%%%%%%%%%%%%%%%%%%%%%%%%

% !===========================================================

% * Classe du document (ClassThesis.cls)
\documentclass{0_Sources/ClassThesis}

% ?-----------------------------------------------------------
% ? SETUP
% ?-----------------------------------------------------------

% * Infos sur la thèse
\Titre{Titre de la thèse}
\Auteur{Prénom NOM}
\Soutenance{jj mois 2026}

% ------------------------------------------------------------

% * Packages (StyleThesis.sty)
\usepackage{0_Sources/StyleThesis}

% ------------------------------------------------------------

% * Metadata du manuscrit
\hypersetup{
    pdfauthor={\theAuteur},
    pdfsubject={Manuscrit de thèse de \theAuteur: \theTitre},
    pdftitle={\theTitre},
    pdfkeywords={Givrage, Turbulence, Transition, Jets, Simulation numérique}
}

% ------------------------------------------------------------

% * Configuration du glossaire
\loadglsentries{1_FrontMatter/Nomenclature}                     % Fichier .tex contenant la définitions des entrées gls
    \setkeys{glslink}{hyper=true}                               % Rend les entrées gls cliquables (ramène à la page de nomenclature)
    \renewcommand*{\glstextformat}[1]{\textcolor{black}{#1}}    % Rend les entrées gls noires

% ------------------------------------------------------------

% * Externalisation de Tikz
\tikzexternaldisable%
%\tikzexternalenable%

% ?-----------------------------------------------------------
% ? CHAPITRES
% ?-----------------------------------------------------------

\includeonly{
% ~ FrontMatter
    {1_FrontMatter/1_Couverture_de_these},
    {1_FrontMatter/2_Seconde_de_couverture},
    {1_FrontMatter/3_Remerciements},
    {1_FrontMatter/4_Nomenclature},
% ~ Chapitres
    {2_Chapitres/0_Introduction},
%
    {2_Chapitres/1_Contexte},
    {2_Chapitres/2_Theorie},
    {2_Chapitres/3_Problematique},
%
    {2_Chapitres/4_Physique},
    {2_Chapitres/5_Numerique},
    {2_Chapitres/6_Experimentale},
%
    {2_Chapitres/7_Resultats},
    {2_Chapitres/8_Discussion},
%
    {2_Chapitres/Conclusion},
    {3_BackMatter/Valorisation},
% ~ Annexes
    {4_Annexes/A_Demonstration},
    {4_Annexes/B_Algorithmes},
% ~ BackMatter
    {3_BackMatter/Resume},
}

% !===========================================================

\begin{document}

% ?-----------------------------------------------------------
% ? FRONTMATTER
% ?-----------------------------------------------------------

\frontmatter%
    \onehalfspacing%                                            % Espacement interligne à 1.5

% * Couverture de thèse
\includepdf{1_FrontMatter/couverture_these}
    \cleardoublepage%

% * Seconde de couverture
%%%%%%%%%%%%%%%%%%%%%%%%%%%%%%%%%%%%%%%%%%%%%%%%%%%
%%                                              %%
%%              SECONDE DE COUVERTURE           %%
%%                                              %%
%%%%%%%%%%%%%%%%%%%%%%%%%%%%%%%%%%%%%%%%%%%%%%%%%%%

% Ici vous n'avez qu'à modifier le verbatim pour la référence de votre manuscrit

\pagestyle{empty}

\vspace*{1cm}

{
    \begin{center}

        {\bfseries En vue de l'obtention du} \\[0.5cm]
        {\Large \bfseries Doctorat de l'Université de Toulouse} \\[0.5cm]
        {\bfseries Délivré par:} {\it l'Institut Supérieur de l'Aéronautique et de l'Espace (ISAE)}\\[1.5cm]
        \rule{\textwidth}{2.5pt}\\[-2ex]
        \rule{\textwidth}{1.2pt}
        {\bfseries \fontfamily{cmr} Présentée et soutenue le} \textit{\theSoutenance} {\bfseries \fontfamily{cmr} par:}\\[0.2cm]
        \settowidth{\RuleWidth}{\Large\fontfamily{cmr}\textbf{\theAuteur}}
        {\Large\textbf{\theAuteur}} \\[0.2cm]
        \vspace{-5pt}
        ~{\color{pdgred}\rule{1.2\RuleWidth}{0.8pt}}\\[0.2cm]
        \vspace{3pt}
        {\Large\bfseries \theTitre}
        \rule{\textwidth}{1.2pt}\\[-1.6ex]
        \rule{\textwidth}{2.5pt}\\[0.5cm]

    \end{center}

    {\bfseries Date:}\\[0.2cm]
    \phantom{\bfseries Date: } \theSoutenance%
}

    \vfill

    Pour citer cette thèse:
    \begin{quote}
        \theAuteur.
        \theTitre.
        ISAE-SUPAERO, Université de Toulouse, \theSoutenance.
    \end{quote}

    \vspace{0.4cm}

    Entrée Bib\TeX{}:
    \begin{quote}
        \begin{verbatim}
@phdthesis{id_reference,
  title        = {Titre de la thèse},
  author       = {Auteur de la thèse},
  year         = {aaaa},
  month        = mmm,
  address      = {Toulouse, Occitanie, France},
  school       = {ISAE-SUPAERO, Universit\'{e} de Toulouse},
  type         = {Th\`{e}se de doctorat}
}
        \end{verbatim}
    \end{quote}

\cleardoublepage%

\null%

\epigraphhead[350]{\large\it{Citation}\par\hfill\textsc{-- Auteur}}

\null%

\restoregeometry%
\pagestyle{fancy}

    \cleardoublepage%

% * Remerciements
%%%%%%%%%%%%%%%%%%%%%%%%%%%%%%%%%%%%%%%%%%%%%%%%%%
%%                                              %%
%%                  REMERCIEMENTS               %%
%%                                              %%
%%%%%%%%%%%%%%%%%%%%%%%%%%%%%%%%%%%%%%%%%%%%%%%%%%
\chapter*{Remerciements}
\mtcaddchapter%
\markboth{Remerciements}{Remerciements}

% Surement les derniers mots de manuscrit que vous écrirez :)
    \cleardoublepage%

% * Table des matières
{\parskip=0pt                                                   % Pas d'interligne
\backrefsetup{disable}                                          % Désactivation du backref le temps de la génération de la ToC
\setcounter{tocdepth}{2}                                        % Profondeur de la ToC (chapter=0, section=1, subsection=2)
\tableofcontents                                                % Génération de la ToC
    \addcontentsline{toc}{chapter}{Table des matières}          % Rajout de la ToC en tant que chapitre (pour apparaitre dans la ToC)
    \cleardoublepage%                                 
\backrefsetup{enable}}                                          % Réactivation du backref

% * Nomenclature
%%%%%%%%%%%%%%%%%%%%%%%%%%%%%%%%%%%%%%%%%%%%%%%%%%
%%                                              %%
%%                Nomenclature                  %%
%%                                              %%
%%%%%%%%%%%%%%%%%%%%%%%%%%%%%%%%%%%%%%%%%%%%%%%%%%

\chapter*{Nomenclature}

\mtcaddchapter%
\addcontentsline{toc}{chapter}{Nomenclature}
\markboth{Nomenclature}{Nomenclature}

\textbf{Sauf mention contraire, les notations utilisées dans ce document sont celles définies ci-dessous.}

\glsfindwidesttoplevelname%
\setglossarystyle{alttree}

% Vous pouvez rajouter des catégories de glossaire ici

{
\parskip=0pt

% Acronymes
\renewcommand*{\glossarysection}[2][]{\section*{Acronymes}}
\printglossary[type=\acronymtype]

% Lettres romanes
\renewcommand*{\glossarysection}[2][]{\section*{Lettres romanes}}
\printglossary[type=roman]

% Lettres grecques
\renewcommand*{\glossarysection}[2][]{\section*{Lettres grecques}}
\printglossary[type=grec]

% Opérateurs et Symboles
\renewcommand*{\glossarysection}[2][]{\section*{Opérateurs et Fonctions}}
\printglossary[type=operator]

}

    \cleardoublepage%

% ?-----------------------------------------------------------
% ? MAINMATTER
% ?-----------------------------------------------------------

\mainmatter%
    \adjustmtc%                                                 % Ajuste les mini-ToC si décallage

% ~-----------------------------------------------------------

% * Introduction
\include{2_Chapitres/0_Introduction}        
    \cleardoublepage%

% ~-----------------------------------------------------------

% * Partie 1 : État de l'art
\epigraphhead[450]{\large\it{Citation}\par\hfill\textsc{-- Auteur}}%
\part{État de l'art}%
\label{part:Etat_de_lart}

% ------------------------------------------------------------

\include{2_Chapitres/1_Contexte}
    \cleardoublepage%
\include{2_Chapitres/2_Theorie}
    \cleardoublepage%
\include{2_Chapitres/3_Problematique}
    \cleardoublepage%

% ~-----------------------------------------------------------

% * Partie 2 : Développements
\epigraphhead[500]{\large\it{Citation}\par\hfill\textsc{-- Auteur}}%
\part{Développements numériques et expérimentaux}%
\label{part:Developpements_numeriques_et_experimentaux}

% ------------------------------------------------------------

\include{2_Chapitres/4_Physique}              
    \cleardoublepage%
\include{2_Chapitres/5_Numerique}
    \cleardoublepage%
\include{2_Chapitres/6_Experimentale}
    \cleardoublepage%

% ~-----------------------------------------------------------

% * Partie 3 : Résultats
\epigraphhead[550]{\large\it{Citation}\par\hfill\textsc{-- Auteur}}%
\part{Résultats et discussions}%
\label{part:Resultats_et_discussions}

% ------------------------------------------------------------

\include{2_Chapitres/7_Resultats}
    \cleardoublepage%
\include{2_Chapitres/8_Discussion}
    \cleardoublepage%

% ~-----------------------------------------------------------

% * Conclusion et valorisation des travaux
\bookmarksetup{startatroot}                                     %TODO
\include{2_Chapitres/Conclusion}
    \cleardoublepage%
\include{3_BackMatter/Valorisation}
    \cleardoublepage%

% ?-----------------------------------------------------------
% ? ANNEXES
% ?-----------------------------------------------------------

\clearpage                                                      %TODO
\appendix
    \renewcommand{\chaptername}{Annexe}                         % Fait apparaitre "Annexe" au lieu de "Chapitre" dans le titre

\epigraphhead[380]{\large\it{Citation}\par\hfill\textsc{-- Auteur}}%
\part*{Annexes}%
\label{part:Annexes}

% ------------------------------------------------------------

\include{4_Annexes/A_Demonstration}
    \cleardoublepage%
\include{4_Annexes/B_Algorithmes}
    \cleardoublepage%

% ?-----------------------------------------------------------
% ? BACKMATTER
% ?-----------------------------------------------------------

\backmatter%
    \bookmarksetup{startatroot}

% ~-----------------------------------------------------------

% * Bibliographie
\bibliographystyle{0_Sources/BiblioThese_abbrv}                 % Fichier de style pour la bibliographie (BiblioThese_abbrv.bst)
\bibliography{3_BackMatter/Bibliographie}                       % Fichier des entrées bibliographique (Bibliographie.bib)
    \addcontentsline{toc}{chapter}{Bibliographie}               % Ajout du chapitre "Bibliographie"
    \cleardoublepage%

% ~-----------------------------------------------------------

\backrefsetup{disable}                                          % Désactivation du backref

% * Liste des figures (ToF)
\cleardoublepage%                                               %TODO
\phantomsection%                                                % Création d'un point d'ancrage hyperlien pour la ToF
\addcontentsline{toc}{chapter}{Liste des figures}               % Rajout de la ToF en tant que chapitre
\listoffigures                                                  % Génération de la ToF
\cleardoublepage%

% * Liste des tableaux (ToT)
\cleardoublepage%                                               %TODO
\phantomsection%                                                % Création d'un point d'ancrage hyperlien pour la ToT
\addcontentsline{toc}{chapter}{Liste des tableaux}              % Rajout de la ToT en tant que chapitre
\listoftables                                                   % Génération de la ToT
\pagestyle{empty}                                               %TODO
\cleardoublepage%

\backrefsetup{enable}                                           % Réactivation du backref

% ~-----------------------------------------------------------

% * Résumé
\cleardoublepage%
\ifodd\value{page}\hbox{}\newpage\fi                            % Fait apparaitre le rémusé en tant que dernière page du manuscrit (4ème de couverture)
\phantomsection%                                                % Création d'un point d'ancrage hyperlien pour le résumé
\addcontentsline{toc}{chapter}{Résumé}                          % Rajout du résumé en tant que chapitre
\include{3_BackMatter/Resume}

% !===========================================================

\end{document}